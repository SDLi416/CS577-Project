\begin{abstract}
Despite significant advancements in the prevention and diagnosis of cardiovascular disease in recent years, heart disease remains the leading cause of adult mortality worldwide. This is partly attributed to the vital role played by blood circulation in facilitating oxygen-carbon dioxide exchange among most organs within the body. Consequently, any abnormality in the heart's pumping capacity can lead to irreversible damage across multiple organs within a short timeframe (3-8 minutes).

However, current human assessment of cardiac function primarily focuses on limited monitoring of cardiac beat cycles and specialized blood tests. The measurement of left ventricular ejection fraction (the ratio between changes in left ventricular end-systolic volume and left ventricular end-diastolic volume) stands as one of the most crucial indicators for evaluating cardiac function. 

But the traditional measurement of the heart's left ventricular ejection fraction (LVEF) by ultrasound instrumentation is not the gold standard, and is hardly the gold standard. Because the measurement of EF by ultrasound is always semi-quantitative, the value of the measurement appears to be objective, but the operator who performs the measurement is subjective. In the clinical process, it is often found that different operators measure the EF of the same patient, but the results are very different, and even the same operator for the same patient EF measurement, but also get different results. So, how can we obtain as accurate an EF as possible to guide the diagnosis and treatment plan during clinical diagnosis and treatment? 
\end{abstract}
