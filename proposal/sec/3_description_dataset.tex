\section{Description of the dataset}

A standard full resting echocardiogram study consists of a series of 50–100 videos and still images visualizing the heart from different angles, locations and image acquisition techniques (two-dimensional images, tissue Doppler images, color Doppler images and others). Each echocardiogram video corresponds to a unique patient and a unique visit. In this dataset, one apical four-chamber two-dimensional greyscale video is extracted from each study. Each video represents a unique individual as the dataset contains 10,030 echocardiography videos from 10,030 unique individuals who underwent echocardiography between 2016 and 2018 as part of clinical care at Stanford Health Care. Videos were randomly split into 7,465, 1,277 and 1,288 patients, respectively, for the training, validation and test sets.

The randomly selected patients in our data have a range of ejection fractions representative of the patient population who visit the echocardiography laboratory. Videos were acquired by skilled sonographers using iE33, Sonos, Acuson SC2000, Epiq 5G or Epiq 7C ultrasound machines and processed images were stored in a Philips Xcelera picture archiving and communication system. Video views were identified through implicit knowledge of view classification in the clinical database by identifying images and videos labelled with measurements done in the corresponding view. For example, apical four-chamber videos were identified by selecting videos from the set of videos in which a sonographer or cardiologist traced left ventricle volumes and labelled these for analysis to calculate ejection fraction. The apical four-chamber view video was thus identified by extracting the Digital Imaging and Communications in Medicine (DICOM) file linked to the measurements of the ventricular volume used to calculate the ejection fraction.

An automated preprocessing workflow was used to remove identifying information and eliminate unintended human labels. Each subsequent video was cropped and masked to remove text, electrocardiogram and respirometer information, and other information outside of the scanning sector. The resulting square images were either 600 × 600 or 768 × 768 pixels depending on the ultrasound machine and down sampled by cubic interpolation using OpenCV into standardized 112 × 112 pixel videos. Videos were spot-checked for quality control, to confirm view classification and to exclude videos with color Doppler.

This research was approved by the Stanford University Institutional Review Board and data privacy review through a standardized workflow by the Center for Artificial Intelligence in Medicine and Imaging (AIMI) and the University Privacy Office. In addition to masking of text, electrocardiogram information and extra data outside of the scanning sector in the video files as described above, the video data of each DICOM file was saved as an AVI file to prevent any leakage of identifying information through public or private DICOM tags. Each video was subsequently manually reviewed by an employee of the Stanford Hospital familiar with imaging data to confirm the absence of any identifying information before public release.
